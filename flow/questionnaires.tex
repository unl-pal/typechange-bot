\documentclass[paper=letter,fontsize=11pt,DIV=14]{scrartcl}

\KOMAoptions{captions=tableheading,bibliography=totoc}

\usepackage[cleanlook,british]{isodate}
\usepackage{amsmath,amsthm,amssymb}
% \usepackage[plextt,lucida]{mycommon}
\usepackage{csquotes}
\usepackage{booktabs,tabularx}
\usepackage{caption,subcaption}
\usepackage[export]{adjustbox}
\usepackage{graphicx}
\usepackage[inline]{enumitem}
\usepackage{float}

\title{Questionnaires for ``Understanding Developers' Addition and Removal of Type Annotations'' (IRB 23988)}
\author{}
\date{}


\usepackage{nameref}
\usepackage[pdfa,colorlinks]{hyperref}
\usepackage{hyperxmp}
\usepackage[nameinlink]{cleveref}

\begin{document}

\maketitle

\section*{Post-Consent Questionnaire}

To answer \enquote{When declaring a variable, why do developers include (or not) a type annotation?}, we ask the following questions:

\begin{quote}
  \begin{enumerate}
  \item When declaring something where a type annotation is optional (such as variables, fields, functions, or methods), which of the following factors influence your decision on whether to include a type annotation? (Select all that apply)
    \begin{itemize}[label=$\square$]
    \item Code Location (scope, etc.)
    \item Complexity of the type annotation
    \item Complexity of the initial value
    \item Compliance with code style requirements
    \item As an aid in documentation
    \item Other, explain briefly below:
    \end{itemize}
  \item Are there situations where you always include a type annotation?
    If so, please list them below, if not, leave blank.
  \item Are there situations where you never include a type annotation?
    If so, please list them below, if not, leave blank.
  % \item TODO: Others?
  \end{enumerate}
\end{quote}


\section*{Post-Change Questionnaire}

To answer \enquote{Why do developers add or remove type annotations?}, we ask one question:

\begin{quote}
  Briefly (1-2 sentences), why did you add/remove type annotation(s) in this code change?
\end{quote}

\end{document}

%%% Local Variables:
%%% mode: LaTeX
%%% TeX-master: t
%%% End:
